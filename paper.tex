\documentclass[11pt]{sty/oecu-thesis}
\usepackage{cite}
\usepackage[dvipdfmx]{graphicx}
\usepackage{sty/subfigure}
\usepackage{subcaption}
\usepackage{sty/lcaption}
\usepackage{times}
\usepackage{url}
\usepackage{amsmath}

% もう少し具体的なタイトルにする。
\title[ナップサック問題における動的計画法による解法の性能調査]{ナップサック問題における動的計画法での性能調査}
\author{小野 遼介}
\date{{令和}\rensuji{元}年\rensuji{12}月\rensuji{16}日}
\学生番号{HT17A027}
\指導教員{久松 潤之 准教授}

% 特別研究の場合はコメントをはずす.卒業研究の場合はコメントアウトする.
%\論文種別{特別研究論文}
\年度{令和元}

\所属{総合情報学部 情報学科}


\begin{document}


\newcommand{\insertfigeps}[2]{%
  \begin{figure}[tb]
    \begin{center}
      \leavevmode
      \includegraphics[width=.80\textwidth]%
      {figure/#1.eps}
      \lcaption{#2}
      \label{fig:#1}
    \end{center}
  \end{figure}}

\newcommand{\insertfigpng}[2]{%
  \begin{figure}[tb]
    \begin{center}
      \leavevmode
      \includegraphics[width=.80\textwidth]%
      {figure/#1.png}
      \lcaption{#2}
      \label{fig:#1}
    \end{center}
  \end{figure}}



%\makeextratitle
\maketitle
\pagenumbering{roman}
\begin{abstract}
  ナップサック問題(KP:Knapsack Problem)はNP困難問題で現実的時間内で最適解を導くことが難しい。
  しかし、近年のコンピュータの性能ならばある程度のサイズの問題ならば最適解を導き出すことができる。
  そこで、動的計画法用いてナップサック問題がどの程度のサイズの問題まで現実的時間内で解くことができるのか調査した。

\keywords % 主な用語
  ナップサック問題(Knapsack Problem)\quad
  NP困難\quad
  動的計画法(DP:Dynamic Programming)\quad
\end{abstract}


\tableofcontents
% 以下の二つは,論文のフォーマットにそっていないが,
% 確認用のためにつける.論文提出時には,コメントアウトする.
\listoffigures
\listoftables
\cleardoublepage

\setcounter{page}{1}
\pagenumbering{arabic}

\chapter{はじめに}
\label{cha:intro}

ナップサック問題(KP:Knapsack Problem)はNP困難問題といわれており、現実的時間内で最適解を導くことが難しいとされている。最適解を求めるアルゴリズムとして全探査、貪欲法、動的計画法(DP:Dynamic Programming)の三つの方法が主流である。ナップサック問題のようなNP困難な問題を解く際には、基本的には近似解を求めるヒューリスティックな手法が一般的に使われることが多い。ヒューリスティックな手法として焼きなまし法と遺伝的アルゴリズムが挙げられる。これらのヒューリスティックな手法を用いれば最適解は求める事は出来ないが最適解に近い近似解を求めることが出来る。
しかし、近年のコンピュータの性能ならばある程度の問題までならば現実的時間内で最適解を導くことができる。そこで、本論文ではナップサック問題を動的計画法を用いて計算し、どの程度のサイズの問題までならば現実的時間内で計算することができるのか調査を行った。

上記でも記述したが最適解を求めるアルゴリズムとして、全探査、貪欲法、動的計画法の三つの方法が挙げられる。本論文では動的計画法を用いる。動的計画法を用いた理由として全探査、貪欲法よりも計算速度が早いということが挙げられる。自作アルゴリズムの動的計画法を実装しナップサック問題の計算を取り組んだ。また、最適解を求める手法である最も基本である全探査プログラムとの比較を行った。

本論文の構成は以下の通りである.
まず,\ref{cha:related} 章ではDPについて.
3章,では提案手法.
4章,ではDPによる解法の性能調査
最後に,\ref{cha:conclusion} 章では,本論文のまとめと今後の課題を述べる.
\chapter{DPとは}
\label{cha:related}
2.1 DP

DPとは、計算効率の良いアルゴリズム構造の一つであり、一度計算した計算式を再利用することで計算効率を上昇させたものである。また、DPは決まったアルゴリズムがある訳ではなく下記の二つを満たしているアルゴリズムを総称してDPと定義される。

・帰納的な関係を利用していること。

・計算結果の記録をしていること。

以上の条件を満たしている時DPといえる。

DPを利用することで計算時間を短縮し最適解を求めることが出来る。またDPを形成する手法としてメモ化や漸化式などの手法が挙げられる。本論文ではメモ化を用いたDPを使ってKPを解く。

2.2 KP

KPとは決められた容積を持つナップサックに価値と容積が決められた品物を入れていき一番価値が高くなる組合せを求める問題である。

例えば、a,b,cの三つの品物とNというナップサックがあったとする。a,b,c,はそれぞれに価値と容積が定められており、またNには容積が定められている。この問題に対しての計算の組合せはa,b,c,a+b,a+c,a+b+cの6通りである。これらの計算の結果容積がN以下であり一番価値の値が大きかった組合せを解とする。全探査の場合ではこれらの計算を全て行うがDPの場合は例えばa+bを一度計算した場合a+b+cの計算ではa+bで求めた計算結果を再利用して計算を行う。分かりやすくいうならばa+b=10という計算結果が出たとしよう。a+b+cではa+bを再利用するので10+cの計算を行うといった手法である。この程度の問題であるならばDPを使わなくても一瞬で計算することが出来るが問題サイズが大きくなると結果として計算速度を上昇させてくれる。

2.3 DPの問題点

DPは計算速度が上昇するが以下の問題点が存在する。

・メモリを大量に使用する。

・普通に計算するより、再利用する計算結果を見つける時の方が遅くなる可能性がある。

・

1つ目の問題点は、メモリを大量に使用するといった点である。int型で2byte、価値と容積で2つ使うため1つの品物につき4byteのメモリを使用する。また、組合せ保存用にも同じ4byte必要で合計8byte必要となる。この時、2^品物の個数*8のメモリが必要になる。今回の実験環境でのメモリ数は31.9GBなので品物の個数31個までしか計算することが出来なかった。品物の個数32個からは2^32*8のメモリを使用することになるのでおおよそ34GBのメモリを要する事になる。しかし、メモリの値段はかなり安い物になっているので問題ではないと考えている。2つ目の問題点は、全探査で計算を行った方がDPよりも早くなるといった問題点である。DPの手法によりメモ化した計算結果を探し出し再利用する方が全探査で行うよりも早くなる可能性がある。しかし、このような問題が起こるのはメモ化したものを最初から目的のメモまでを参照場合のみ起こりうるので本論文でのDPではそのような問題がないようにアルゴリズムを作成した。

\chapter{提案手法}
\label{cha:xxx}
DPは漸化式を用いた手法とメモ化を用いた手法の2つがある。本論文ではメモ化を用いてプログラムを作成した。提案手法は計算の組合せを二進数に置き換える方法で、例えばa,b,cの品物が存在したとき、まず001から111までの二進数の表を作成する。1桁目にa、2桁目にb、3桁目にcを対応させる。001ならばa、010ならばb、100ならばcとする。これにより111ならばa+b+cと表すことが出来る。毎回の計算結果を二進数の表に入れていく。
\insertfigeps{sample-eps}{eps 画像の貼り付けの例}
\insertfigpng{sample-png}{png 画像の貼り付けの例}
\chapter{DP解法での性能調査}
\label{cha:xxx}
このファイルは、ファイル名や章タイトル、そして、label を適宜書き換えること。

\insertfigeps{sample-eps}{eps 画像の貼り付けの例}
\insertfigpng{sample-png}{png 画像の貼り付けの例}
\chapter{まとめと今後の課題}
\label{cha:conclusion}

 本稿では,~~.
 
 今後の課題としては、~~.
\acknowledgment
本研究と本論文を終えるにあたり、御指導、御教授を頂いた久松潤之准教授に
深く感謝致します。また、学生生活を通じて、基礎的な学問、学問に取り組む
姿勢を御教授頂いた、登尾啓史教授、升谷保博教
授、渡邊郁教授、南角茂樹教授、鴻巣敏之教授、北嶋暁教授、大西克彦
准教授、小枝正直准教授に深く感謝致します。

本研究期間中、本研究に対する貴重な御意見、御協力を頂きました久松研究室
の皆様に心から御礼申し上げます。


\bibliographystyle{junsrt}
\bibliography{bib/myrefs}


\end{document}
