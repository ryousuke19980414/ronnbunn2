\chapter{はじめに}
\label{cha:intro}

ナップサック問題(KP:Knapsack Problem)はNP困難問題といわれており、現実的時間内で最適解を導くことが難しいとされている。最適解を求めるアルゴリズムとして全探査、貪欲法、動的計画法(DP:Dynamic Programming)の三つの方法が主流である。ナップサック問題のようなNP困難な問題を解く際には、基本的には近似解を求めるヒューリスティックな手法が一般的に使われることが多い。ヒューリスティックな手法として焼きなまし法と遺伝的アルゴリズムが挙げられる。これらのヒューリスティックな手法を用いれば最適解は求める事は出来ないが最適解に近い近似解を求めることが出来る。
しかし、近年のコンピュータの性能ならばある程度の問題までならば現実的時間内で最適解を導くことができる。そこで、本論文ではナップサック問題を動的計画法を用いて計算し、どの程度のサイズの問題までならば現実的時間内で計算することができるのか調査を行った。

上記でも記述したが最適解を求めるアルゴリズムとして、全探査、貪欲法、動的計画法の三つの方法が挙げられる。本論文では動的計画法を用いる。動的計画法を用いた理由として全探査、貪欲法よりも計算速度が早いということが挙げられる。自作アルゴリズムの動的計画法を実装しナップサック問題の計算を取り組んだ。また、最適解を求める手法である最も基本である全探査プログラムとの比較を行った。

本論文の構成は以下の通りである.
まず,\ref{cha:related} 章ではDPについて.
3章,では提案手法.
4章,ではDPによる解法の性能調査
最後に,\ref{cha:conclusion} 章では,本論文のまとめと今後の課題を述べる.