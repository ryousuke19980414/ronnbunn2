\chapter{提案手法}
\label{cha:xxx}
DPは漸化式を用いた手法とメモ化を用いた手法の2つがある。本論文ではメモ化を用いてプログラムを作成した。提案手法は計算の組合せを二進数に置き換える方法で、例えばa,b,cの品物が存在したとき、まず001から111までの二進数の表を作成する。1桁目にa、2桁目にb、3桁目にcを対応させる。001ならばa、010ならばb、100ならばcとする。これにより111ならばa+b+cと表すことが出来る。毎回の計算結果を二進数の表に入れていく。この表に入れた結果を毎回計算するときに参照できるようにした。また、参照する際は

\insertfigeps{sample-eps}{eps 画像の貼り付けの例}
\insertfigpng{sample-png}{png 画像の貼り付けの例}